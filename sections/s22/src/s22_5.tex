\documentclass{spisok-article}

\usepackage{amsmath}
\usepackage{amssymb}
\usepackage{amsthm}
\usepackage{amscd}
\usepackage{bm}
\usepackage{upref}
\usepackage{bbold}

\title{Исследование задачи размещения Ролса с прямоугольной метрикой на отрезке прямой\thanks{ Работа выполнена при финансовой поддержке РГНФ, проект \textnumero 16-02-00059.}
}

\author{Кривулин Н. К.,
	д.ф.-м.н., профессор кафедры статистического моделирования СПбГУ, nkk@math.spbu.ru\\
	Плотников П. В.,
	аспирант кафедры статистического моделирования СПбГУ,
	pavplot@gmail.com
}

\usepackage[all]{xy}

\setcounter{tocdepth}{2}

\newtheorem{theorem}{Теорема}
\newtheorem{lemma}{Лемма}
\newtheorem{corollary}{Следствие}

\begin{document}

\maketitle

\begin{abstract}
	В статье рассматривается задача размещения точечного объекта на плоскости с прямоугольной метрикой в терминах тропической математики с ограничениями на область размещения в виде отрезка прямой. Предложено полное решение задачи и рассмотрен численный пример. 
\end{abstract}

\section{Введение}

Тропическая математика -- раздел прикладной математики, занимающийся изучением полуколец с идемпотентным сложением \cite{Maslov1994Idempotent,Golan2003Semirings,Krivulin2009Methods,Butkovic2010Maxlinear}. Модели и методы тропической математики находят применение для решения различных оптимизационных задач в технике, экономике и на производстве. Краткий обзор таких задач можно найти, например, в работе \cite{Krivulin2014Tropical}. В частности, методы тропической оптимизации успешно применяются при решении ряда задач размещения в пространстве, включая минимаксные задачи размещения с чебышевской и прямоугольной метрикой \cite{Krivulin2011Analgebraic,Krivulin2012Anewalgebraic,Krivulin2015Onanalgebraic}.

Минимаксная задача размещения одиночного объекта на плоскости, которую также называют задачей Ролса или задачей посыльного, в прямоугольной метрике имеет при отсутствии ограничений на допустимую область размещения известное геометрическое решение \cite{Elzinga1972Geometrical,Francis1972Ageometrical}. В работе \cite{Krivulin2015Onanalgebraic} предложено полное алгебраическое решение такой задачи в явном виде в замкнутой форме. 

В настоящей статье задача Ролса рассматривается при условии, что допустимое множество размещения имеет вид отрезка произвольной прямой. Сначала вводятся основные определения тропической математики, затем формулируется задача размещения на плоскости с прямоугольной метрикой в терминах тропической математики. Предлагается новое решение задачи, которое позволяет записать результат в более компактной форме, чем решение, полученное в \cite{Krivulin2016Usingtrop}. В заключение, приведен численный пример с графическими иллюстрациями. 

\section{Элементы тропической математики}

Рассмотрим числовое множество $\mathbb{X}$, замкнутое относительно операций сложения $\oplus$ и умножения $\otimes$. Заданное на $\mathbb{X}$ при помощи этих операций коммутативное полукольцо с нулем $\mathbb{0}$ и единицей $\mathbb{1}$ обозначим через $\langle\mathbb{X},\mathbb{0},\mathbb{1},\oplus,\otimes\rangle$ . Будем считать, что сложение обладает свойством идемпотентности, т.е. для любого числа $x\in\mathbb{X}$ выполняется $x\oplus x = x$, а умножение -- обратимо, т.е. для каждого $x\ne\mathbb{0}$ существует обратный элемент $x^{-1}$ такой, что $x\otimes x^{-1}=\mathbb{1}$. В силу того, что $\langle \mathbb{X}\setminus\{\mathbb{0}\}, \mathbb{1}, \otimes \rangle$ образует коммутативную группу по умножению, введенное полукольцо, обычно называется идемпотентным полуполем.

Степень числа с целым показателем вводится стандартным образом. Для любого ненулевого числа $x\in\mathbb{X}$ и натурального числа $n$ определим $x^{0}=\mathbb{1}$, $x^{n} =x\otimes x^{n-1}$, $x^{-n}=(x^{-1})^{n}$ и $\mathbb{0}^{n}=\mathbb{0}$. Будем считать, что операция возведения в целую степень может быть распространена на случай произвольного показателя степени. В частности, для любого $x<\mathbb{1}$ при $n=0$ положим $x^{1/n}=\mathbb{0}$.
Далее знак умножения $\otimes$ в алгебраических выражениях, как обычно, опускается.

Примером идемпотентного полуполя являются вещественное полуполе
$
\mathbb{R}_{\max,+}=\langle\mathbb{R}\cup\{-\infty\},-\infty,0,\max,+\rangle,
$
где $\mathbb{R}$ -- множество вещественных чисел.

Нулевым элементом в $\mathbb{R}_{\max,+}$ является $-\infty$, единичным -- число $0$. Каждому числу $x\in\mathbb{R}$ в этом полуполе сопоставляется обратный элемент $x^{-1}$, равный $-x$ в обычной алгебре. Для любой пары чисел $x,y\in\mathbb{R}$ определена степень $x^{y}$, значение которой соответствует арифметическому произведению $xy$. 

\section{Приложения к задачам размещения на прямой}

Рассмотрим минимаксную задачу размещения точечного объекта на плоскости с прямоугольной метрикой и ограничениями на область размещения. Пусть задан набор исходных объектов и некоторое допустимое множество $S\subset\mathbb{R}^{2}$. Требуется разместить новый объект на множестве $S$, относительно уже имеющихся так, чтобы расстояние в прямоугольной метрике, от него до самой удаленного объекта, было минимальным.

Пусть на плоскости $\mathbb{R}^{2}$ имеются два вектора $\bm{x}=(x_{1},x_{2})^{T}$ и $\bm{y}=(y_{1},y_{2})^{T}$. Расстояние между этими векторами в прямоугольной метрике вычисляется по формуле
$
\rho(\bm{x},\bm{y})
=
|x_{1}-y_{1}|+|x_{2}-y_{2}|.
$

Рассмотрим набор точек $\bm{r}_{i}=(r_{1i},r_{2i})^{T}\in\mathbb{R}^{2}$ и чисел $w_{i}\in\mathbb{R}$, заданных для всех $i=1,\ldots,m$. Для произвольного вектора $\bm{x}=(x_{1},x_{2})^{T}$ введем функцию
$$
\phi(\bm{x})
=
\max_{1\leq i \leq m}(\rho(\bm{r}_{i},\bm{x})+w_{i})
=
\max_{1\leq i \leq m}(|r_{1i}-x_{1}|+|r_{2i}-x_{2}|+w_{i})
=
\phi(x_{1},x_{2}),
$$
которая определяет максимальное по всем $i$ расстояние в прямоугольной метрике от точки $\bm{x}$ до точки $\bm{r}_{i}$ с учетом дополнительного слагаемого $w_{i}$.

Зададим множество размещения на плоскости в виде отрезка прямой
\begin{equation}
S
=
\{(x_{1},x_{2})^{T}\ | \quad f\leq x_{1}\leq g,\ x_{2}=kx_{1}+q\}.
\label{E-S-fx1g-x2kx1q}
\end{equation}

Минимаксная задача размещения, или задача Ролса, состоит в том, чтобы найти все векторы $\bm{x} \in S$, на которых достигается минимум
\begin{equation}
\begin{aligned}
&
\min_{\bm{x}\in S}
&&
\phi(\bm{x}).
\end{aligned}
\label{P-minphix}
\end{equation}

В терминах идемпотентного полуполя $\mathbb{R}_{\max,+}$ расстояние между двумя векторами в прямоугольной метрике можно представить в форме
$
\rho(\bm{x},\bm{y})
=
(x_{1}^{-1}y_{1}\oplus y_{1}^{-1}x_{1})
(x_{2}^{-1}y_{2}\oplus y_{2}^{-1}x_{2}).
$
Тогда целевая функция задачи \eqref{P-minphix} записывается следующим образом:
\begin{equation*}
\phi(x_{1},x_{2})
=
\bigoplus_{i=1}^{m}w_{i}
(x_{1}^{-1}r_{1i}\oplus r_{1i}^{-1}x_{1})
(x_{2}^{-1}r_{2i}\oplus r_{2i}^{-1}x_{2}),
\end{equation*}
а множество размещения \eqref{E-S-fx1g-x2kx1q} принимает вид
\begin{equation*}
S
=
\{(x_{1},x_{2})^{T}\ | \quad f\leq x_{1}\leq g,\ x_{2}=x_{1}^k q\}.
\end{equation*}

\subsection{Размещение на произвольной прямой}

Будем рассматривать задачу размещения на множестве $S$. Пусть заданы числа $f,g,q,k\in\mathbb{R}$ при условии $f\leq g$. 

Следующий результат обеспечивает полное решение задачи в явном виде.

\begin{lemma}
	Введем обозначения
\begin{equation*}
\begin{aligned}
a
=
\bigoplus_{i=1}^{m}w_{i}r_{1i}r_{2i}^{-1}q, \quad
b
=
\bigoplus_{i=1}^{m}w_{i}r_{1i}^{-1}r_{2i}q^{-1},
\\
c
=
\bigoplus_{i=1}^{m}w_{i}r_{1i}r_{2i}q^{-1}, \quad
d
=
\bigoplus_{i=1}^{m}w_{i}r_{1i}^{-1}r_{2i}^{-1}q.
\end{aligned}
\end{equation*}
	Тогда справедливы следующие утверждения:
	\begin{enumerate}\renewcommand\labelenumi{\textup{\theenumi)}}
		\item
		если $k<-1$ или $k>1$, то минимум в задаче \eqref{P-minphix} при условии \eqref{E-S-fx1g-x2kx1q} равен
		\begin{multline*}
		\mu
		=
		a^{1/2}b^{1/2}
		\oplus
		a^{(k+1)/2k}c^{(k-1)/2k}
		\oplus
		b^{(k+1)/2k}d^{(k-1)/2k}
		\oplus
		c^{1/2}d^{1/2}
		\oplus
		\\
		\oplus
		a(f^{-(k-1)} \oplus g^{-(k-1)})^{-1} 
		\oplus
		b(f^{k-1} \oplus g^{k-1})^{-1}
		\oplus
		\\
		\oplus
		c(f^{k+1} \oplus g^{k+1})^{-1}
		\oplus
		d(f^{-(k+1)} \oplus g^{-(k+1)})^{-1}
		\end{multline*}
		и достигается тогда и только тогда, когда
		\begin{multline*}
		x_{1}
		=
		(((\mu a^{-1})^{-1/(k-1)} \oplus (\mu^{-1}b)^{-1/(k-1)})^{-1}\oplus
		\\
		\oplus
		((\mu^{-1}c)^{-1/(k+1)} \oplus (\mu d^{-1})^{-1/(k+1)})^{-1}\oplus f)^{1-\alpha}\\
		(((\mu^{-1}a)^{-1/(k-1)} \oplus (\mu b^{-1})^{1/(k-1)} )^{-1}\oplus
		\\
		\oplus
		((\mu c^{-1})^{1/(k+1)} \oplus (\mu^{-1}d)^{-1/(k+1)})^{-1}\oplus g^{-1})^{-\alpha},
		\end{multline*}
		\begin{equation*}
		x_{2}
		=
		qx_{1}^{k};
		\end{equation*}
		\item
		если $-1\leq k\leq 1$, то минимум равен
		\begin{multline*}
		\mu
		=
		a^{1/2}b^{1/2}
		\oplus
		a^{(k+1)/2}d^{-(k-1)/2}
		\oplus
		b^{(k+1)/2}c^{-(k-1)/2}
		\oplus
		c^{1/2}d^{1/2}
		\oplus
		\\
		\oplus
		ag^{k-1}
		\oplus
		bf^{-(k-1)}
		\oplus
		cg^{-(k+1)}
		\oplus
		df^{k+1}
		\end{multline*}
		и достигается тогда и только тогда, когда
		\begin{multline*}
		x_{1}
		=
		((\mu a^{-1})^{1/(k-1)}\oplus(\mu^{-1}c)^{1/(k+1)}\oplus f)^{1-\alpha}\\
		((\mu b^{-1})^{1/(k-1)}\oplus(\mu^{-1}d)^{1/(k+1)}\oplus g^{-1})^{-\alpha},
		\end{multline*}
		\begin{equation*}
		x_{2}
		=
		qx_{1}^{k},
		\end{equation*}
		где $\alpha$ -- любое число, удовлетворяющее условию $0\leq\alpha\leq1$.
	\end{enumerate}
\end{lemma}

При использовании обычных обозначений найденное решение описывается так:
\begin{corollary}
	Введем обозначения
	\begin{align*}
	a
	&=
	\max_{1\leq i \leq m}(w_{i}-r_{1i}+r_{2i}+q),
	&
	b
	&=
	\max_{1\leq i \leq m}(w_{i}+r_{1i}-r_{2i}-q),
	\\
	c
	&=
	\max_{1\leq i \leq m}(w_{i}+r_{1i}+r_{2i}-q),
	&
	d
	&=
	\max_{1\leq i \leq m}(w_{i}-r_{1i}-r_{2i}+q).
	\end{align*}
	Тогда справедливы следующие утверждения:
	\begin{enumerate}\renewcommand\labelenumi{\textup{\theenumi)}}
		\item
		если $k<-1$ или $k>1$, то минимум в задаче \eqref{P-minphix} при условии \eqref{E-S-fx1g-x2kx1q} равен
	\begin{multline*}
	\mu
	=
	\max\bigg(\frac{a+b}{2},\frac{a(k+1)+c(k-1)}{2k},\frac{b(k+1)+d(k-1)}{2k},\frac{c+d}{2},
	\\
	a+\min((k-1)f,(k-1)g),b-\max((k-1)f,(k-1)g),
	\\
	c-\max((k+1)f,(k+1)g) ,d+\min((k+1)f,(k+1)g)\bigg)
	\end{multline*}
	и достигается тогда и только тогда, когда
	\begin{multline*}
	x_{1}
	=
	(1-\alpha)\max\left(\min\left(\frac{\mu-a}{k-1},\frac{b-\mu}{k-1}\right),\min\left(\frac{c-\mu}{k+1},\frac{\mu-d}{k+1}\right),f\right)
	\\
	-\alpha\max\left(\min\left(\frac{a-\mu}{k-1},\frac{\mu-b}{k-1}\right),\min\left(\frac{\mu-c}{k+1},\frac{d-\mu}{k+1}\right),-g\right),
	\end{multline*}
    \begin{equation*}
	x_{2}
	=
	kx_{1}+q;
	\end{equation*}
	\item
		если $-1\leq k\leq 1$, то минимум равен
	\begin{multline*}
	\mu
	=
	\max\bigg(\frac{a+b}{2},
	\frac{a(k+1)-d(k-1)}{2},
	\frac{b(k+1)-c(k-1)}{2},
	\frac{c+d}{2},
	\\
	a+(k-1)g,
	b-(k-1)f,
	c-(k+1)g,
	d+(k+1)f\bigg)
	\end{multline*}
	и достигается тогда и только тогда, когда
	\begin{equation*}
	x_{1}
	=
	(1-\alpha)\max\left(\frac{\mu-a}{k-1},\frac{c-\mu}{k+1},f\right)
	-
	\alpha\max\left(\frac{\mu-b}{k-1},\frac{d-\mu}{k+1},-g\right),
	\end{equation*}
	\begin{equation*}
	x_{2}
	=
	kx_{1}+q,
	\end{equation*}
		где $\alpha$ -- любое число, удовлетворяющее условию $0\leq\alpha\leq1$.
	\end{enumerate}
\end{corollary}

\section{Иллюстрация результатов на примерах}

Рассмотрим числовой пример. Пусть задано множество точек 
\begin{equation*}
\bm{r}_{1} = (1,\ 7)^{T}, \quad \bm{r}_{2} = (3,\ 3)^{T}, \quad \bm{r}_{3} = (4,\ 6)^{T}, \quad \bm{r}_{4} = (5,\ 3)^{T},
\end{equation*}
\begin{equation*}
\bm{r}_{5} = (7,\ 2)^{T}, \quad \bm{r}_{6} = (9,\ 1)^{T}, \quad \bm{r}_{7} = (9,\ 9)^{T}.
\end{equation*}

Сначала рассмотрим решение задачи размещения \eqref{P-minphix}, в которой допустимое множество представляет собой отрезок, заданный в виде 
\begin{equation*}
S_{1}
=
\{(x_{1},x_{2})^{T}\ | \quad 5\leq x_{1}\leq 8,\ x_{2}=-2x_{1}+19\}.
\end{equation*}
Решение такой задачи приведено на Рис.~1, где выделенный жирным отрезок прямой представляет множество решений задачи без ограничений, отрезок обычной толщины -- допустимое множество размещения, а точка на пересечении -- решение задачи с ограничениями.

Перейдем к рассмотрению случая, в котором допустимая область размещения задана следующим образом:
\begin{equation*}
S_{2}
=
\{(x_{1},x_{2})^{T}\ | \quad 4\leq x_{1}\leq 7,\ x_{2}=2x_{1}-6\}.
\end{equation*}

В этом случае, как показано на Рис.~2, допустимая область размещения не пересекается с множеством решений задачи без ограничений. Решение исходной задачи изображено точкой, лежащей на отрезке $S_{2}$.

\begin{flushleft}
	\begin{picture}(320,200)
	
	\put(0,45){\vector(1,0){140}} %ось x
	\put(5,40){\vector(0,1){140}} %ось y
	\put(137.5,34){$x_1$} %подпись ось х
	\put(-9,178.5){$x_2$} %подпись ось y
	\put(-6,34){$0$} %центр 0
	
	%горизонтальные
	\put(25,42){\line(0,1){3}} %насечки
	\put(23,33.5){$2$}
	\put(45,42){\line(0,1){3}} %насечки
	\put(43,33.5){$4$}
	\put(65,42){\line(0,1){3}} %насечки
	\put(63,33.5){$6$}
	\put(85,42){\line(0,1){3}} %насечки
	\put(83,33.5){$8$}
	\put(105,42){\line(0,1){3}} %насечки
	\put(101,33.5){$10$}
	\put(125,42){\line(0,1){3}} %насечки
	\put(121,33.5){$12$}
	
	%вертикальные
	\put(2,65){\line(1,0){3}} %насечки
	\put(-7.5,65){$2$}
	\put(2,85){\line(1,0){3}} %насечки
	\put(-7.5,85){$4$}
	\put(2,105){\line(1,0){3}} %насечки
	\put(-7.5,105){$6$}
	\put(2,125){\line(1,0){3}} %насечки
	\put(-7.5,125){$8$}
	\put(2,145){\line(1,0){3}} %насечки
	\put(-12.5,145){$10$}
	\put(2,165){\line(1,0){3}} %насечки
	\put(-12.5,165){$12$}
	
	\put(55,135,5){\line(1,-2){30}}
	\put(72,102){\circle{4}} %решение
	
	\put(15,115){\circle*{3}}
	\put(45,105){\circle*{3}}
	\put(95,55){\circle*{3}}
	\put(35,75){\circle*{3}}
	\put(75,65){\circle*{3}}
	\put(55,75){\circle*{3}}
	\put(95,135){\circle*{3}}
	
	\put(65,96){\line(1,1){10}} %линия
	\put(65,95.5){\line(1,1){10}} %линия
	\put(65,95){\line(1,1){10}} %линия
	\put(65,94.5){\line(1,1){10}} %линия
	\put(65,94){\line(1,1){10}} %линия
	
	\put(-5,2){\textsf{Рис.1 Размещение на отрезке $S_{1}$.}} % подпись к рисунку
	
	%второй рисунок 
	\put(165,45){\vector(1,0){150}} %ось x 
	\put(170,40){\vector(0,1){140}} %ось y
	\put(312.5,34){$x_1$} %подпись ось х
	\put(156,178.5){$x_2$} %подпись ось y
	\put(159,34){$0$} %центр 0
	
	%горизонтальные
	\put(190,42){\line(0,1){3}} %насечки
	\put(188,33.5){$2$}
	\put(210,42){\line(0,1){3}} %насечки
	\put(208,33.5){$4$}
	\put(230,42){\line(0,1){3}} %насечки
	\put(228,33.5){$6$}
	\put(250,42){\line(0,1){3}} %насечки
	\put(248,33.5){$8$}
	\put(270,42){\line(0,1){3}} %насечки
	\put(266,33.5){$10$}
	\put(290,42){\line(0,1){3}} %насечки
	\put(286,33.5){$12$}
	
	%вертикальные
	\put(167,65){\line(1,0){3}} %насечки
	\put(157.5,65){$2$}
	\put(167,85){\line(1,0){3}} %насечки
	\put(157.5,85){$4$}
	\put(167,105){\line(1,0){3}} %насечки
	\put(157.5,105){$6$}
	\put(167,125){\line(1,0){3}} %насечки
	\put(157.5,125){$8$}
	\put(167,145){\line(1,0){3}} %насечки
	\put(152.5,145){$10$}
	\put(167,165){\line(1,0){3}} %насечки
	\put(152.5,165){$12$}
	
	\put(210,65){\line(1,2){30}} %граница+
	\put(225,95){\circle{4}} %решение

	\put(180,115){\circle*{3}}
	\put(200,75){\circle*{3}}
	\put(210,105){\circle*{3}}
	\put(220,75){\circle*{3}}
	\put(240,65){\circle*{3}}
	\put(260,55){\circle*{3}}
	\put(260,135){\circle*{3}}

	\put(230,96.5){\line(1,1){10}} %линия
	\put(230,96){\line(1,1){10}} %линия
	\put(230,95.5){\line(1,1){10}} %линия
	\put(230,95){\line(1,1){10}} %линия
	\put(230,94.5){\line(1,1){10}} %линия
	
	\put(160,2){\textsf{Рис.2 Размещение на отрезке $S_{2}$.}} %% подпись к рисунку
	\label{fig2}
	
	\end{picture}
\end{flushleft}


\begin{thebibliography}{99}
	
	\bibitem{Maslov1994Idempotent}
	\textit{Маслов~В.~П., Колокольцов~В.~Н.}
	Идемпотентный анализ и его применение в оптимальном управлении.
	М.: Физматлит, 1994.
	144~с.
	
	\bibitem{Golan2003Semirings}
	\textit{Golan~J.~S.}
	Semirings and Affine Equations Over Them.
	New York: Springer, 2003.
	Vol.~556 of Mathematics and Its Applications.
	256~p.
	
	\bibitem{Krivulin2009Methods}
	\textit{Кривулин~Н.~К.}
	Методы идемпотентной алгебры в задачах моделирования и анализа сложных систем.
	СПб.: Изд-во С.-Петерб. ун-та., 2009.
	256~с.
	
	\bibitem{Butkovic2010Maxlinear}
	\textit{Butkovi\v{c}~P.}
	Max-linear Systems. Springer Monographs in Mathematics.
	London: Springer, 2010.
	272~p.
	
	\bibitem{Krivulin2014Tropical}
	\textit{Krivulin~N.}
	Tropical optimization problems //
	Advances in Economics and Optimization: Collected Scientific Studies Dedicated to the Memory of L.~V.~Kantorovich /
	Ed. by L.~A.~Petrosyan, J.~V.~Romanovsky, D.~W.~K.~Yeung.
	New York: Nova Science Publ., 2014.
	P.~195--214.
	
	\bibitem{Krivulin2011Analgebraic}
	\textit{Krivulin~N.}
	 An algebraic approach to multidimensional minimax location problems with Chebyshev distance // 
	 WSEAS Trans. Math.
	 2011.
	 Vol.~10, N~6.
	 P.~191-–200.	
	 
	 \bibitem{Krivulin2012Anewalgebraic}
	 \textit{Krivulin~N.}
	A new algebraic solution to multidimensional minimax location problems with Chebyshev distance // 
	 WSEAS Trans. Math.
	 2012.
	 Vol.~11, N~7.
	 P.~605–-614.
	
	\bibitem{Elzinga1972Geometrical}
	\textit{Elzinga~J., Hearn~D.~W.}
	Geometrical solutions for some minimax location problems //
	Transport. Sci.
	1972.
	Vol.~6, N~4.
	P.~379--394.
	
	\bibitem{Francis1972Ageometrical}
	\textit{Francis~R.~L.}
	A geometrical solution procedure for a rectilinear distance minimax location problem //
	AIIE Trans.
	1972.
	Vol.~4, N~4.
	P.~328--332.
	
	\bibitem{Krivulin2015Onanalgebraic}
	\textit{Krivulin~N.~K., Plotnikov~P.~V.}
	On an algebraic solution of the Rawls location problem in the plane with rectilinear metric //
	Vestnik St.~Petersburg Univ. Math.
	2015.
	Vol.~48, N~2.
	P.~75-81.
	
	\bibitem{Krivulin2016Usingtrop}
	\textit{Krivulin~N.~K., Plotnikov~P.~V.}
	Using tropical optimization to solve minimax location
	problems with a rectilinear metric on the line //
	Vestnik St.~Petersburg Univ. Math.
	2016.
	Vol.~49, N~4.
	P.~340-349.	
	
\end{thebibliography}

\end{document}
